\chapter{Baselines Lab Details} \label{apx:BaselinesLab}
Appendix \ref{apx:BaselinesLab} gives some detailed information on baselines lab. We start with detailed information on used python libraries in Section \ref{sec:BLLibraries}, continue with details on how the lab can be configured with config files in Section \ref{sec:BLConfigFiles} and finish with a description of the various command line arguments in Section \ref{sec:BLCommandLine}.

\section{Python Libraries} \label{sec:BLLibraries}
Baselines Lab uses the Python distribution Anaconda \cite{anaconda}. The lab depends on a number of Python packages. While some of these packages will be installed automatically, some of them must be installed manually. Individual packages might not be available through the Anaconda distribution itself and must be installed using the python package manager PIP. Table \ref{tab:PackageRequirements} lists all packages which are required to run baselines lab.

\begin{table}[ht]
    \begin{center}
        \small
        \begin{threeparttable}
            \begin{tabular}{|p{4cm}|R{1.5cm}|}
                \hline
                \multicolumn{1}{|c|}{Package} & \multicolumn{1}{c|}{Version} \\
                \hline
                gym & 0.15.4 \\
                imageio & 2.6.1 \\
                matplotlib & 3.1.2 \\
                numpy & 1.17.4 \\
                opencv-python & 4.1.2.30 \\
                optuna & 1.0.0 \\
                pandas & 0.25.3 \\
                Pillow & 7.0.0 \\
                PyYAML & 5.2 \\
                stable-baselines & 2.10.0 \\
                tensorflow \tnote{1} & 1.14.0 \\
                mpi4py & 3.0.3 \\
                \hline
            \end{tabular}
            \begin{tablenotes} \footnotesize
                \item[1] Uses the GPU version of Tensorflow.
            \end{tablenotes}
        \end{threeparttable}
    \end{center}
    \vspace*{-1em}
    \caption[Python Package Requirements]{Python Package Requirements} \label{tab:PackageRequirements}
    \vspace*{-0.8em}
\end{table}


When installing Python packages, many dependencies will be installed automatically. To ensure test reproducibility, we also included a complete list of all packages installed in our virtual test environment in Table \ref{tab:AdditionalPackages}.

\begin{table}[ht]
    \begin{center}
        \small
        \begin{tabular}{c c}
            \begin{tabular}{|p{4cm}|R{1.75cm}|}
                \hline
                \multicolumn{1}{|c|}{Package} & \multicolumn{1}{c|}{Version} \\
                \hline
                absl-py & 0.8.1 \\
                alembic & 1.3.3 \\ 
                astor & 0.8.0 \\
                atari-py & 0.2.6 \\
                certifi & 2020.4.5.1 \\
                cliff & 2.18.0 \\
                cloudpickle & 1.2.2 \\
                cmd2 & 0.8.9 \\
                colorlog & 4.1.0 \\
                cycler & 0.10.0 \\
                future & 0.18.2 \\
                gast & 0.3.2 \\
                google-pasta & 0.1.8 \\
                grpcio & 1.16.1 \\
%                gym & 0.15.4 \\
                h5py & 2.9.0 \\
%                imageio & 2.6.1 \\
                joblib & 0.14.1 \\
                Keras-Applications & 1.0.8 \\
                Keras-Preprocessing & 1.1.0 \\
                kiwisolver & 1.1.0 \\
                Mako & 1.1.1 \\
                Markdown & 3.1.1 \\
                MarkupSafe & 1.1.1 \\
%                matplotlib & 3.1.2 \\
                mkl-fft & 1.0.15 \\
                mkl-random & 1.1.0 \\

                \hline
            \end{tabular} &
            \begin{tabular}{|p{4cm}|R{1.75cm}|}
                \hline
                \multicolumn{1}{|c|}{Package} & \multicolumn{1}{c|}{Version} \\
                \hline

                mkl-service & 2.3.0 \\
                mpi4py & 3.0.3 \\
%                numpy & 1.17.4 \\
                olefile & 0.46 \\
%                opencv-python & 4.1.2.30 \\
%                optuna & 1.0.0 \\
%                pandas & 0.25.3 \\
                pbr & 5.4.4 \\
%                Pillow & 7.0.0 \\
                prettytable & 0.7.2 \\
                protobuf & 3.11.2 \\
                pyglet & 1.3.2 \\
                pyparsing & 2.4.5 \\
                pyperclip & 1.7.0 \\
                python-dateutil & 2.8.1 \\
                python-editor & 1.0.4 \\
                pytz & 2019.3 \\
%                PyYAML & 5.2 \\
                scipy & 1.3.2 \\
                six & 1.13.0 \\
                SQLAlchemy & 1.3.13 \\
%                stable-baselines & 2.10.0 \\
                stevedore & 1.31.0 \\
                tensorboard & 1.14.0 \\ 
%                tensorflow \tnote{1} & 1.14.0 \\
                tensorflow-estimator & 1.14.0 \\
                termcolor & 1.1.0 \\
                tqdm & 4.42.1 \\
                typing & 3.7.4.1 \\
                wcwidth & 0.1.8 \\
                Werkzeug & 0.16.0 \\
                wrapt & 1.11.2 \\
                \hline
            \end{tabular} \\
        \end{tabular}  
    \end{center}
    %\vspace*{-1em}
    \caption[Additional Python Packages]{List of all additional Python packages in the virtual Python environment used for all experiments.} \label{tab:AdditionalPackages}
\end{table}

\section{Functions and Configuration Files} \label{sec:BLConfigFiles}
We already talked about lab configuration files in Section \ref{sec:blFunctions}. In this section we want to list all Baselines Lab specific keywords for lab configuration files and give a few more general examples. 

\begin{table}[ht]
    \begin{center}
        \small
        \bgroup
        \def\arraystretch{1.25}
        \begin{tabular}{|>{\ttfamily}c|p{0.74\textwidth}|}
            \hline
            \normalfont{Keyword} & \multicolumn{1}{c|}{Description} \\
            \hhline{|=|=|}
            name & Specifies the type of RL algorithm to use during training. Each learning algorithm which is available in Stable Baselines can be used in Baselines Lab and chosen by specifying its lowercase abbreviation (e.g. a2c for actor-critic). Note that there exist two implementations for the PPO algorithm which can be chosen via "ppo1" or "ppo2". \\
            tensorboard\_log & Specifies weather or not logging should be enabled. \\
            policy & Defines the policy. The policy keyword itself is the parent of a \texttt{name} keyword which specifies a registered policy class. All additional keywords will get passed to the policy class. \\
            trained\_agent & Specifies if training should begin based on a pretrained model. The trained agent can either be \textit{best} or \textit{last} to automatically load a model from the last run saved in the specified log directory, or a complete path to a specific directory containing a model checkpoints directory. \\
            \hline
        \end{tabular}
        \egroup
    \end{center}
    %\vspace*{-1em}
    \caption[Configuration File Algorithm Keyword]{Description of the Baselines Lab specific keywords for the algorithm section.} \label{tab:AlgorithmKeywords}
\end{table}

\paragraph{Algorithm.} We begin with \textit{algorithm} specific keywords which define and parameterize the RL algorithm used during training. The \texttt{algorithm} keyword is therefore mandatory for each configuration file. We listed Baselines Lab specific parameters in Table \ref{tab:AlgorithmKeywords}. Additionally, all parameters available for the individual RL algorithm implementations can be specified as children of the algorithm keyword and will be directly forwarded. A list of these parameters can be found in the Stable Baselines documentation \cite{stable-baselines-docs}. 

\begin{table}[!ht]
    \begin{center}
        \small
        \bgroup
        \def\arraystretch{1.25}
        \begin{tabular}{|>{\ttfamily}c|p{0.75\textwidth}|}
            \hline
            \normalfont{Keyword} & \multicolumn{1}{c|}{Description} \\
            \hhline{|=|=|}
            scale & Weather or not to scale the inputs by $x \mapsto x / 255$ \\
            extractor\_arch & Architecture for the CNN extractor layers. The structure of the CNN is given as a list of tuples in the form of ("conv", n\_filters, filter\_size, stride) for a convolutional layer or ("pool", size, stride) for a pooling layer. \\
            mlp\_arch & Architecture for the MLP layers which get their input from the CNN extractor. May be a list of integer values where each value defines the number of neurons in a fully connected layer. The last element may be a dictionary with the entries \textit{pi} and \textit{vf} which both may contain lists of integers to define a diverging network for the policy network and the value function. \\
            extractor\_act & Defines the activation function for the extractor network. \\
            mlp\_act & Defines the activation function for the MLP network. \\
            \hline
        \end{tabular}
        \egroup
    \end{center}
    %\vspace*{-1em}
    \caption[Configuration of the GeneralCnnPolicy]{Description of the Baselines Lab specific keywords for the policy class GeneralCnnPolicy.} \label{tab:NetworkKeywords}
\end{table}

Baselines Lab also defines a new network type for fully configurable CNN networks. While Stable Baselines only offers to fully configure an MLP network, Baselines Lab introduces a new network class which is fully customizable. To use the new network type, we use the policy type \textit{GeneralCnnPolicy} and can then use the keywords as shown in Table \ref{tab:NetworkKeywords} to customize the network. 

\begin{figure}[hp]
    \lstinputlisting[language=yaml]{figures/baselines_details/network.yml}
    \vspace*{-1em}
    \caption{Basic GeneralCnnPolicy configuration.}
    \label{fig:BasicNetworkConfig}
%    \vspace*{-1em}
\end{figure}

We also included an example configuration which is given in Figure \ref{fig:BasicNetworkConfig}. The configuration constructs a network with three convolutional layers with 32, 64 and 64 filters respectively which is directly connected to a fully-connected layer with 512 neurons (similar to the network used by Huang et al. see \ref{sec:TDDRL}). The network then diverges into two ends where the policy network and the value network both have a dedicated fully-connected layer with 256 neurons. Leaky ReLU activations are used for both the CNN extractor and the MLP network.


\paragraph{Environment.} The environment can be configured using the \texttt{env} keyword. Since every training or enjoy session needs an environment, the env keyword is also mandatory for each lab configuration file. The env keyword also specifies how observations are processed before they are passed to the RL algorithm and may pass values to the environment itself. Table \ref{tab:EnvironmentKeywords} gives a complete overview over all available keywords. All keywords which do not match any lab specific keyword will be directly passed to the environment.

\begin{table}[hp]
    \begin{center}
        \small
        \bgroup
        \def\arraystretch{1.25}
        \begin{tabular}{|>{\ttfamily}c|p{0.75\textwidth}|}
            \hline
            \normalfont{Keyword} & \multicolumn{1}{c|}{Description} \\
            \hhline{|=|=|}
            name & Specifies the name of the environment. The name must be registered in the OpenAI gym registry. \\
            n\_envs & Specifies the number of parallel environments to use. May be one for a single environment only. \\
            multiprocessing & If the given number of environments is greater than one, a subprocess will be created for each individual environment to make use of multicore processors. This can be disabled by setting multiprocessing to false. \\
            normalize & Specifies if a normalization wrapper should be used. The normalization wrapper may be parameterized by specifying additional subkeys which are documented in \cite{stable-baselines-docs}. \\
            frame\_stack & Specifies if a frame stacking wrapper should be used. On default four frames are used, but the number can be changed by specifying an \texttt{n\_stack} subkey. \\
            scale & Specifies if a scaling wrapper should be used. The scaling wrapper transforms image observations in the form of $x \mapsto x/255$. \\
            curiosity & Specifies if a RND curiosity wrapper should be used. The curiosity wrapper will be added after the normalization wrapper. The wrapper specific parameters can be configured to automatically match the algorithm parameters by setting the subkey \texttt{auto\_params} to true. Additional wrapper specific parameters can be found in Appendix TODO. \\
            wrappers & Specifies a list of environment wrappers which should be used. The wrappers are given by complete class names similar to a python import statement. Arguments for each wrapper can be passed, by creating child keywords. \\
            \hline
        \end{tabular}
        \egroup
    \end{center}
    \vspace*{-1em}
    \caption[Configuration File Env Keyword]{Description of the Baselines Lab specific keywords for the environment section.} \label{tab:EnvironmentKeywords}
    \vspace*{-1em}
\end{table}

\paragraph{Meta.} The \textit{meta} keyword defines a number of general training and program parameters, most important the number of training steps \texttt{n\_timesteps} and the log directory \texttt{log\_dir}. A complete list of all available keywords and their meaning can be found in Table \ref{tab:MetaKeywords}.

\begin{table}[hp]
    \begin{center}
        \small
        \bgroup
        \def\arraystretch{1.25}
        \begin{tabular}{|>{\ttfamily}c|p{0.7\textwidth}|}
            \hline
            \normalfont{Keyword} & \multicolumn{1}{c|}{Description} \\
            \hhline{|=|=|}
            n\_timesteps & Total number of training steps for the model. \\
            log\_dir & Location of the log directory. Baselines Lab will automatically create a subfolder with a timestamp at the given location for each individual run. \\
            seed & The global random seed. \\
            save\_interval & Number of timesteps between each model checkpoint. \\
            n\_keep & Total number of checkpoint to store (defaults to 5). If a new checkpoint is created the oldest checkpoint may be deleted if more than n\_keep checkpoints are already saved. \\
            keep\_best & Sets weather or not to always save the best model. Also enables or disables model evaluation during training. \\
            n\_eval\_episodes & Number of episodes during training evaluation. Defaults to 32. \\
            n\_trials & Number of experiment repetitions. If n\_trials is greater than one, the experiment will be repeated several times with the same parameters. \\
            plot & Sets weather or not plots should be automatically generated at the end of the training from the tensorboard log data. \\
            record\_images & Sets weather or not to save images from the environment to the tensorboard log during training. \\
            \hline
        \end{tabular}
        \egroup
    \end{center}
    \caption[Configuration File Meta Keyword]{Description of the Baselines Lab specific keywords for the meta section.} \label{tab:MetaKeywords}
\end{table}


\paragraph{Search.} To parameterize automated hyperparametersearch, a dedicated section must be created with the keyword \texttt{search}. In this section we can define the sampler and pruner methods as well as general parameters (e.g. how many trials should be performed). We included a description for all available keywords in Table \ref{tab:SearchKeywords}. Pruners and samplers can be created by either stating their name only, or by creating a subkey named \texttt{method} and then adding additional subkeys to parameterize the sampler or pruner. Parameters for the samplers and pruners can be found in the Optuna documentation at \cite{optuna-docs}.


\begin{table}[hp]
    \begin{center}
        \small
        \bgroup
        \def\arraystretch{1.25}
        \begin{tabular}{|>{\ttfamily}c|p{0.7\textwidth}|}
            \hline
            \normalfont{Keyword} & \multicolumn{1}{c|}{Description} \\
            \hhline{|=|=|}
            n\_trials & Total number of trials to run to find optimal hyperparameters. \\
            n\_timesteps & Total number of training steps during hyperparametersearch. \\
            n\_evaluations & Total number of evaluations for each trial. The trial may be pruned after each evaluation, if its performance is worse than the performance of previous trials depending on the pruner. \\
            eval\_method & Evaluation method to use. May be \textit{fast}, \textit{normal} or \textit{slow}. See Section \ref{sec:blFunctions} for details. \\
            n\_test\_episodes & Number of episodes to perform during evaluation. \\
            deterministic & Sets weather or not to use deterministic actions during evaluation. \\
            n\_jobs & Sets the number of trials to run in parallel (defaults to one). \\
            \hline
        \end{tabular}
        \egroup
    \end{center}
    \caption[Configuration File Search Keyword]{Description of the Baselines Lab specific keywords for the search section.} \label{tab:SearchKeywords}
\end{table}


The search section is also used to define which parameters should be included in the hyperparametersearch. Depending on the used RL algorithm, a set of parameters will be automatically included with a default range of possible parameters. To set parameters to a fixed value, these parameters must be defined in the regular \texttt{algorithm} section. To add new parameters to the hyperparametersearch or update the default range, we can add a new \texttt{algorithm} or \texttt{env} keyword as subkey of the \texttt{search keyword}. Instead of assigning values to the keyword, we then create subkeys defining the possible values. An example can be seen in figure \ref{fig:BasicSearchConfig}. All sampling methods supported by Optuna can be also used in Baselines Lab, by defining the method with the \texttt{method} keyword (see Optuna documentation for more info \cite{optuna-docs}). For all range based sampling methods, we can then define their lowest value and their highest value with the keywords \texttt{low} and \texttt{high}. The \textit{categorical} method receives a list from a keyword called \texttt{choices}.


\section{Command Line Arguments} \label{sec:BLCommandLine}
Baselines Lab offers a number of command line arguments which control the general lab behavior. We already featured the most important arguments which specify the lab mode in Section \ref{sec:blFunctions}. All lab modes additionally require the definition of a lab config file. Independent of the lab mode there are three optional arguments which can be specified:

\begin{itemize}
    \item \texttt{-{}-verbose VERBOSE} Specifies the verbosity level. The level is given as an integer value which directly corresponds to the Python logging level (e.g. 10 for \texttt{DEBUG}).
    \item \texttt{-{}-mail MAIL} Provides an option to specify a mail address which will receive a notification if training is done, or failed. This option requires the linux program \textit{mailx} to be installed.
    \item \texttt{-{}-ignore-errors} Is a flag, which is only available if the lab is started with a batch of configuration files. If ignore-errors is set, execution will continue, even if one configuration file produces an error. 
\end{itemize}

Depending on the lab mode, there are mode specific command line arguments. We listed each available argument in Table \ref{tab:CommandLineArguments}.

\begin{table}[hp]
    \begin{center}
        \small
        \bgroup
        \def\arraystretch{1.25}%  1 is the default, change whatever you need
        \begin{tabular}{|c|c|p{0.55\textwidth}|}
            \hline
            Mode & Argument & \multicolumn{1}{c|}{Description} \\
            \hhline{|=|=|=|}
            \texttt{Train} & \texttt{-{}-trial TRIAL} & If training is resumed from a previous run with multiple trials, Trial 0 is used on default. This behavior can be changed, by specifying a specific trial. \\
            \hline
            \multirow{15}{*}{\texttt{Enjoy}} & \texttt{-{}-type TYPE} & Specifies which type of checkpoint to load (best or last). Defaults to best. \\
            & \texttt{-{}-checkpoint-path PATH} & Path to a log directory to load the model from. Defaults to the log directory specified in the given config file. \\
            & \texttt{-{}-video} & If set, a video is created from the rendered environment output and saved to the log directory. \\
            & \texttt{-{}-obs-video} & If set, a video is created from the observations generated by the environment and saved to the log directory. \\
            & \texttt{-{}-stochastic} & If set, the model will use stochastic actions instead of deterministic actions. \\
            & \texttt{-{}-evaluate EPISODES} & Activates model evaluation in normal evaluation mode during enjoy replay for given number of episodes. Evaluation results will be saved in the log directory. \\
            & \texttt{-{}-strict} & If set, evaluation will only use a single environment (equals slow evaluation mode). \\
            & \texttt{-{}-trial TRIAL} & The number of the trial to load from the log directory. Defaults to trial zero. \\
            & \texttt{-{}-plot} & If set, plots will be generated from the training tensorboard data and saved in the model's log directory.\\
            \hline
            \texttt{Search} & \texttt{-{}-plot} & Sets weather or not to plot the distribution of choosen hyperparameters. \\
            \hline
        \end{tabular}
        \egroup
    \end{center}
    \vspace*{-1em}
    \caption[Mode Specific Command Line Arguments]{Description of lab mode specific command line arguments.} \label{tab:CommandLineArguments}
    \vspace*{-2em}
\end{table}

\section{Random Network Distillation Module}
We already described how we implemented RND curiosity reward in Section \ref{sec:blRND}. In this section we want to list all parameters for our RND reward wrapper. Each of these parameters can be configured using lab config files. We included a complete list in Table \ref{tab:RNDParameters}. The parameters \textit{gamma}, \textit{learning\_rate}, \textit{train\_freq} and \textit{buffer\_size} can be automatically set to match the parameters used for the RL algorithm, by setting the key \texttt{auto\_params: true} in the configuration file under \texttt{curiosity}. All default parameters are set to match the original implementation as close as possible.

\begin{table}[ht]
    \begin{center}
        \small
        \bgroup
        \def\arraystretch{1.25}
        \begin{tabular}{|>{\ttfamily}c|p{0.65\textwidth}|}
            \hline
            \normalfont{Parameter} & \multicolumn{1}{c|}{Description} \\
            \hhline{|=|=|}
            intrinsic\_reward\_weight & Weight for the intrinsic reward. \\
            buffer\_size & Size of the internal replay buffer used to train the predictor network. \\
            train\_freq & Number of steps between each training of the predictor network. \\
            gradient\_steps & Number of training epochs. \\
            batch\_size & Number of samples to use during each training epoch. Samples are randomly drawn from the replay buffer. \\
            learning\_starts & Step at which the predictor will be trained for the first time. \\
            filter\_end\_of\_episode & Weather or not the end of episode signal should be filtered and not propagated to the RL algorithm. \\
            filter\_reward & Weather or not to filter extrinsic reward. \\
            norm\_obs & Weather or not to normalize observations for the target and predictor networks. \\
            norm\_ext\_reward & Weather or not to norm the extrinsic reward. \\
            gamma & Gamma for the reward normalization. \\
            learning\_rate & The predictor learning rate. \\
            training & Weather or not the predictor should be trained. \\
            monitor & Weather or not to create curiosity specific Tensorboard logs. \\
            \hline
        \end{tabular}
        \egroup
    \end{center}
    \caption[Parameters for the RND Wrapper]{Description of the parameters for the RND wrapper.} \label{tab:RNDParameters}
\end{table}